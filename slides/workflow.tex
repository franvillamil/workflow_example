\documentclass[aspectratio=43]{beamer}
% \documentclass[aspectratio=169]{beamer}

% Title --------------------------------------------
\title[]{\huge Quantitative Research Workflow}
\author[]{Francisco Villamil}
\date[]{UC3M -- \today}

\input{beamer_preamble.tex}

\begin{document}
% ====================================================

% ----------------------------------------------------
\begin{frame}
  \titlepage
\end{frame}
% ----------------------------------------------------

% ----------------------------------------------------
\begin{frame}
\frametitle{Reason to think about a workflow}
\centering

\begin{itemize}
  \item[1.] \textbf{Automate stuff:} you spend a lot of time on the computer so make it work for you
  \item[2.] \textbf{Avoid errors:} we should not trust ourselves
\end{itemize}

\end{frame}
% ----------------------------------------------------

% ----------------------------------------------------
\begin{frame}
\frametitle{Problems}
\centering

\begin{itemize}[<+->] \small
\setbeamercovered{transparent}
  \item I revise a Word document too many times and end up with \texttt{final.docx}, \texttt{final2.docx}, \texttt{finalnov23.docx}, \texttt{finalFINAL.docx}
  \item After four months, I go back to a data project and have one 5000-thousand R file that I completely do not understand anything of it
  \item I run an old R file and suddenly it doesn't run because a file is missing - and I don't know where it is
  \item The project is ready and instead of using data since 1991, I want to use data since 1989: do I have to run everything again?
  \item I have a book or a dissertation (or a MA thesis) ready, with 60+ tables, and after it's ready, I found a little mistake that just changes the second decimal in the analyses tables - good luck updating all those tables
  \item Mistake in the code because it doesn't tell me where it's wrong (Real: \href{https://journals.sagepub.com/doi/10.1177/20531680221126454}{journals.sagepub.com/doi/10.1177/20531680221126454})
\end{itemize}

\end{frame}
% ----------------------------------------------------

% ----------------------------------------------------
\begin{frame}
\frametitle{Principles}
\centering

\begin{enumerate}
  \item Using computers
  \begin{itemize}
    \item Use \textit{plain text} files as much as possible
    \item Customize your work tool
    \item Use a code/text editor and make it \textit{yours}
    \item Learn how to use the Terminal (unix commands) and automate
    \item Use version control (git)
  \end{itemize}
  \item Coding and empirical projects
  \begin{itemize}
    \item Separate code in specific tasks, be tidy
    \item Integrate different parts of same project (R, tex...)
    \item \textbf{Automate output} (tables, plots ...)
    \item Use \textbf{functions} (automate code), i.e. do not do the same thing twice
    \item Checks and warnings in code
    \item (Optional: consider using \textit{Makefile})
  \end{itemize}
  \begin{itemize}

  \end{itemize}
\end{itemize}

\end{frame}
% ----------------------------------------------------

% ----------------------------------------------------
\begin{frame}
\frametitle{Some resources}
\centering

\begin{itemize}
  \item Hadley Wickham's \href{http://adv-r.had.co.nz/Style.html}{R Style guide} (and the whole \href{http://adv-r.had.co.nz/}{Advanced R book} later on)
  \item Software Carpentry's lessons: \href{https://software-carpentry.org/lessons/}{https://software-carpentry.org/lessons/}
  \begin{itemize}
    \item Especially \href{https://swcarpentry.github.io/shell-novice/}{Unix Shell} and \href{https://swcarpentry.github.io/git-novice/}{Version Control with Git}
  \end{itemize}
  \item Kieran Healy's \textit{The Plain Person’s Guide to Plain Text Social Science}: \href{https://plain-text.co/}{https://plain-text.co/}
  \begin{itemize}
    \item Although \texttt{emacs} is perhaps a bit too hardcore
  \end{itemize}
  \item The best Git course I know is this: \href{https://gitexercises.fracz.com/}{https://gitexercises.fracz.com/}
  \item MIT's \textit{The Missing Semester of Your CS Education}: \href{https://missing.csail.mit.edu/}{https://missing.csail.mit.edu/}
\end{itemize}

\end{frame}
% ----------------------------------------------------

\section{Using computers}

% ----------------------------------------------------
\begin{frame}
\frametitle{Plain text}
\centering

\begin{itemize}
  \item What's plain text?
  \item Quicker and easier to work with
  \item Cross-platform and does not depend on proprietary software
  \item Much better for the things you want to do
  \begin{itemize}
    \item You can use version control on it
    \item Closer to how machines work it - so easier for whatever related to machines (e.g. syncing two computers) $\rightarrow$ \href{https://github.com/franvillamil/configfiles}{example1}, \href{https://github.com/franvillamil/sublime_settings}{example2}
    \item It's a base ingredient you can convert into whatever (e.g. with \texttt{R}, \texttt{LaTeX}, etc)
  \end{itemize}
\end{itemize}

\end{frame}
% ----------------------------------------------------

% ----------------------------------------------------
\begin{frame}
\frametitle{Customizing your computer}
\centering

\begin{itemize}
  \item \href{https://franvillamil.github.io/posts/setup_macos.html}{https://franvillamil.github.io/posts/setup_macos.html}
  \item \href{https://github.com/franvillamil/templates}{https://github.com/franvillamil/templates}
  \item \href{https://github.com/franvillamil/configfiles}{https://github.com/franvillamil/configfiles}
  \item Examples: mdtopdf/docxtopdf, baserepos, Spectacle, ...
\end{itemize}

\end{frame}
% ----------------------------------------------------

% ----------------------------------------------------
\begin{frame}
\frametitle{Code editor}
\centering

\begin{itemize}
  \item Choose and get used to some code editor
  \begin{itemize}
    \item You're probably using the editor in \texttt{RStudio}, that's fine, but there are reasons to use better and more general tools
  \end{itemize}
  \item You can customize these so suit your needs, e.g.:
  \begin{itemize}
    \item Edit \& run languages you use (\texttt{R}, \texttt{Latex}, whatever)
    \item Small stuff that saves time, like snippets
    \item Navigate a project
    \item And much more complicated stuff we're not going to talk about and that I do not know so much about
  \end{itemize}
  \item I use Sublime Text: \href{https://www.sublimetext.com/}{https://www.sublimetext.com/}
  \item Anyway, \textbf{don't use MS Word} (as much as possible)
\end{itemize}

\end{frame}
% ----------------------------------------------------

% ----------------------------------------------------
\begin{frame}
\frametitle{Using the command line}
\centering

\begin{itemize}
  \item Think of it as the language to communicate with the OS
  \item No need that you become a computer wizard, but I personally think it pays off to learn a little bit
  \item Why?
  \begin{itemize}
    \item Automate stuff in the computer (e.g. from updating local files to converting .docx into pdf)
    \item Navigate and work with files faster
    \item Version control, installing stuff, solving issues
    \item Virtual machines
  \end{itemize}
  \item \textbf{Note:} Unix/Mac vs Windows
\end{itemize}

\end{frame}
% ----------------------------------------------------

% ----------------------------------------------------
\begin{frame}
\frametitle{Version control (Git)}
\centering

\begin{itemize}[<+->]
  \item \texttt{final1.docx}, \texttt{finalfinal.docx}... but in a proper way
  \item You want to keep control of all versions of a file, something like MS Word's `Tracked changes' but just much better
  \begin{itemize}
    \item Keep a time machine of all versions of a file
    \item Allow collaboration between different people (or between two computers)
  \end{itemize}
  \item There is more than one system, but most people use \texttt{Git} (and Github)
\end{itemize}

\end{frame}
% ----------------------------------------------------

% ----------------------------------------------------
\begin{frame}
\frametitle{Version control}
\centering

\includegraphics[width = \textwidth]{img/gitesquema}

\end{frame}
% ----------------------------------------------------

% ----------------------------------------------------
\begin{frame}
\frametitle{Version control}
\centering

\includegraphics[width = \textwidth]{img/git-basic}

\end{frame}
% ----------------------------------------------------

% ----------------------------------------------------
\begin{frame}
\frametitle{Version control - a note}
\centering

\begin{itemize}
  \item Version control works \textbf{much} better if you work with other people who also use version control, which is often not the case (at least not mine)
  \item Yet, there are two advantages to use it in my view:
  \begin{itemize}
    \item Obvious one: keep older versions of a file
    \item If you work with two computers, perhaps Google Drive/Dropbox do not work that well
    \item Virtual machines (e.g. Google Cloud Computing, Amazon Web Services)
  \end{itemize}
\end{itemize}

\end{frame}
% ----------------------------------------------------

\section{Coding better and organizing data projects}

% ----------------------------------------------------
\begin{frame}
\frametitle{Coding projects: tasks as folders}
\centering

\begin{itemize}
  \item This applies especially to the R part of projects
  \item Do not create one huge R code file, use different files for different tasks
  \item You want to do the same with the folder structure
  \begin{itemize}
    \item \textbf{Especially with R output!}
  \end{itemize}
  \item (Optionally, you can use \textit{Makefile}, but it can be problematic)
\end{itemize}

\end{frame}
% ----------------------------------------------------

% ----------------------------------------------------
\begin{frame}
\frametitle{Coding projects: tasks as folders}
\centering

\includegraphics[width = \textwidth]{img/workflow_vox_streets}

\end{frame}
% ----------------------------------------------------

% ----------------------------------------------------
\begin{frame}
\frametitle{Coding projects extra: filenaming}
\centering

\begin{itemize}
  \item Not a lot of things here, but just think about how you name files or folders
  \item[1.] Do not use spaces
  \item[2.] Ideally use some standards (e.g. \texttt{1_clean-file.R}, \texttt{2_merge.R} ...)
  \item \textbf{{\color{red}{DON'T:}}} \texttt{First file.csv}, \textbf{definitely not:} \texttt{Datos educación.csv}
  \begin{itemize}
    \item good: \texttt{first_file.csv}, \texttt{datos_educacion.csv} etc
  \end{itemize}
\end{itemize}

\end{frame}
% ----------------------------------------------------

% ----------------------------------------------------
\begin{frame}
\frametitle{Coding projects, extra: integrate}
\centering

\begin{itemize}
  \item Put together \texttt{Latex} and \texttt{R} parts (and/or Python, etc)
  \item If you organize the folder as I said before, it's pretty much solved
  \item Overleaf example
  \item What you get with this? Avoid mistakes, makes your life easier...
\end{itemize}

\end{frame}
% ----------------------------------------------------

% ----------------------------------------------------
\begin{frame}
\frametitle{Coding projects, extra: Makefile}
\centering

\begin{itemize}
  \item Example
  \item \href{https://makefiletutorial.com/}{https://makefiletutorial.com/}
\end{itemize}

\end{frame}
% ----------------------------------------------------

% ----------------------------------------------------
\begin{frame}
\frametitle{Writing code I: automate via functions}
\centering



\end{frame}
% ----------------------------------------------------

% ----------------------------------------------------
\begin{frame}
\frametitle{Writing code II: write checks and warnings}
\centering

\begin{itemize}
  \item As you write code, always write checks using \texttt{stop()} or \texttt{warning()}, e.g.:
  \begin{itemize}
    \item if a new data frame is built from merging, what should be the number of rows in the final df? or columns?
    \item should two objects be identical?
    \item do we have duplicated values by some ID?
    \item do you expect \texttt{``145``} or \texttt{145} (character vs integer)?
    \item ...
    \item[]
  \end{itemize}
\end{itemize}

\end{frame}
% ----------------------------------------------------

% ----------------------------------------------------
\imageframe{img/stop1}
% ----------------------------------------------------

% ----------------------------------------------------
\imageframe{img/stop2}
% ----------------------------------------------------

% ----------------------------------------------------
\begin{frame}
\frametitle{Writing code II: write checks and warnings}
\centering

\begin{itemize}
\item Also try to minimize errors, e.g. that you have visuals of real output, e.g.:
\item[1.] I use \texttt{print()} all the time to show length of stuff, number of missing data, etc
\item[2.] \texttt{modelsummary} vs \texttt{stargazer} example
  \begin{itemize}
    \item \href{https://journals.sagepub.com/doi/10.1177/20531680221126454}{journals.sagepub.com/doi/10.1177/20531680221126454}
    \item \href{https://github.com/franvillamil/streets_vox/blob/master/robust/rob.R}{github.com/franvillamil/streets\_vox/blob/master/robust/rob.R}
  \end{itemize}
\end{itemize}

\end{frame}
% ----------------------------------------------------

%\appendix
%\renewcommand{\theframenumber}{A\arabic{framenumber}}
%\renewcommand{\insertframenumber}{A\arabic{framenumber}}

% ====================================================
\end{document}
